\documentclass[twocolumn]{aa}

\usepackage{graphicx}
\usepackage{amsmath,amsfonts,amssymb}
\usepackage{txfonts}
\usepackage{color}
\usepackage{natbib}
\usepackage{float}
\usepackage{dblfloatfix}
\usepackage{afterpage}
\usepackage{ifthen}
\usepackage[morefloats=12]{morefloats}
\usepackage{placeins}
\usepackage{multicol}
\bibpunct{(}{)}{;}{a}{}{,}
\usepackage[switch]{lineno}
\definecolor{linkcolor}{rgb}{0.6,0,0}
\definecolor{citecolor}{rgb}{0,0,0.75}
\definecolor{urlcolor}{rgb}{0.12,0.46,0.7}
\usepackage[breaklinks, colorlinks, urlcolor=urlcolor,
    linkcolor=linkcolor,citecolor=citecolor,pdfencoding=auto]{hyperref}
\hypersetup{linktocpage}


%Planck style file, to be used with A&A style to produce Planck papers for publication.
%
% version 28 September 2010 --- useful macros --- CRL
% version 17 October 2010   --- first cut at important instrument values, from Daniele Mennella and
%                               Francois Bouchet, 13 October 2010 --- CRL
% version 18 October 2010   --- LFI FWHM changed to one value per feed, rather than M & S separately
%                               LFI FWHM uncertainties added for individual feeds.  Corrections made
%                               to LFI values. --- Andrea Zacchei
% version 24 October 2010   --- added to and corrected definitions.  No changes made to instrument
%                               quantities. --- CRL 
% version 31 October 2010   --- added definition of \muKHz. --- CRL
%
% version 15 November 2010  --- fixed conflict with aa.cls in definition of \endtable
%                               by naming the command below "\endPlancktable".  See section
%                               13.16 of the Style Guide.
%
% version 06 December 2010  --- Set up names with and without units.
%                               Add \allearlypapers command to ensure that all early papers are
%                               included in the reference list.
%                               Define macro for the name of the 4He JT cooler.
%
% version 07 December 2010  --- removed extraneous "planck2011-1.2" entry in \allearlypapers
%
% version 12 December 2010  --- added \endPlancktablewide command to set tablenotes to the full
%                               page width in the \begin{table*}...\end{table*} environment when
%                               the ``twocolumn'' option is specified in the \documentclass command.
%                               (It would be more elegant to extract the appropriate width from the
%                               aa.cls system at the time of execution, but that is buried more
%                               deeply in the system than I investigated.)
%
% version 05 January 2011   --- added unit \MJysr.  HFI performance values updated per FRB email
%                               01/05/2011 02:38-0800, and Brendan Crill email 01/05/2011 18:08 -0800.
%
% version 06 January 2011   --- changed \scriptscriptstyle primes to \scriptstyle, to better match the
%                               tx fonts used by A&A.
%
% version 07 January 2011   --- modified \allearlypapers to correspond with final early paper list.  
%                               Fixed 545 GHz center frequency.
%
% version 07 January 2011b  --- changed LFI white-noise sensitivity numbers to correct problem with units
%
% version 05 July 2011      --- added \Msol and \Lsol to get the symbols for solar mass and luminosity.
%                               Deleted previous definitions of \solar and \sol, which were equivalent
%                               to the new \Msol.
%
% version 16 August 2011    --- changed comments on \endPlancktable and \endPlancktablewide for clarity
%
% version 11 September 2011 --- changed definition of \tablenote to make footnote labels italic, as per A\&A
%
% version 26 April 2011     --- changed definition of \Planck to agree with what is said in the Style Guide (!)
%
% version 04 Dec 2013       --- included 2013 results references
%
% version 17 Jan 2014       --- included fix to bibtex file v4.3, i.e. \providecommand{\sorthelp}[1]{}
%
% version 26 Jul 2014       --- fixed incompatibility problem with aa.cls v8.0 and v8.2.  v8.2 should now be used
%                               for all Planck papers.
%                           --- fixed problem in definition of "\all2013resultspapers" that introduced a blanck
%                               into the reference to p06b.
%                           --- removed all the parameter definition stuff at the end.  We weren't using it, and
%                               it took up a lot of space.
%
% version 28 Jan 2015       --- added "\alltwentyfiftennresultspapers" and corrected "\all2013resultspapers" to
%                               "\all20thirteenresultspapers",
%
% version 29 Jan 2022       --- added all necessary additional parameters, macros and units for this paper (BP XIV)
% author: K. J. Andersen
%
% Usage:  after the \documentclass[traditabstract]{aa} command in the La\TeX\ input file,
%         add this command:      \input Planck.tex


\def\setsymbol#1#2{\expandafter\def\csname #1\endcsname{#2}}
\def\getsymbol#1{\csname #1\endcsname}

%-----------------------------------------------------------------------
% Planck
%-----------------------------------------------------------------------
\def\Planck{\textit{Planck}}

%-----------------------------------------------------------------------
% The Planck Helium-4 JT cooler
%-----------------------------------------------------------------------
\def\HeJT{$^4$He-JT}

%-----------------------------------------------------------------------
% To include all Planck Early Results papers in the reference lists
%-----------------------------------------------------------------------
\def\allearlypapers{\nocite{planck2011-1.1, planck2011-1.3, planck2011-1.4, planck2011-1.5, planck2011-1.6, planck2011-1.7, planck2011-1.10, planck2011-1.10sup, planck2011-5.1a, planck2011-5.1b, planck2011-5.2a, planck2011-5.2b, planck2011-5.2c, planck2011-6.1, planck2011-6.2, planck2011-6.3a, planck2011-6.4a, planck2011-6.4b, planck2011-6.6, planck2011-7.0, planck2011-7.2, planck2011-7.3, planck2011-7.7a, planck2011-7.7b, planck2011-7.12, planck2011-7.13}}

%-----------------------------------------------------------------------
% To include all Planck 2013 Results papers in the reference lists
%-----------------------------------------------------------------------
\def\alltwentythirteenresultspapers{\nocite{planck2013-p01, planck2013-p02, planck2013-p02a, planck2013-p02d, planck2013-p02b, planck2013-p03, planck2013-p03c, planck2013-p03f, planck2013-p03d, planck2013-p03e, planck2013-p01a, planck2013-p06, planck2013-p03a, planck2013-pip88, planck2013-p08, planck2013-p11, planck2013-p12, planck2013-p13, planck2013-p14, planck2013-p15, planck2013-p05b, planck2013-p17, planck2013-p09, planck2013-p09a, planck2013-p20, planck2013-p19, planck2013-pipaberration, planck2013-p05, planck2013-p05a, planck2013-pip56, planck2013-p06b, planck2013-p01a}}

%-----------------------------------------------------------------------
% To include all Planck 2015 Results papers in the reference lists
%-----------------------------------------------------------------------
\def\alltwentyfifteenresultspapers{\nocite{planck2014-a01, planck2014-a03, planck2014-a04, planck2014-a05, planck2014-a06, planck2014-a07, planck2014-a08, planck2014-a09, planck2014-a11, planck2014-a12, planck2014-a13, planck2014-a14, planck2014-a15, planck2014-a16, planck2014-a17, planck2014-a18, planck2014-a19, planck2014-a20, planck2014-a22, planck2014-a24, planck2014-a26, planck2014-a28, planck2014-a29, planck2014-a30, planck2014-a31, planck2014-a35, planck2014-a36, planck2014-a37, planck2014-ES}}

%-----------------------------------------------------------------------
% Tables
%-----------------------------------------------------------------------
\newbox\tablebox    \newdimen\tablewidth
\def\leaderfil{\leaders\hbox to 5pt{\hss.\hss}\hfil}
%
% use the following definition of \endPlancktable for ApJ style notes to tables, set to the 
%         width of the table
% \def\endPlancktable{\tablewidth=\wd\tablebox 
%
% use the following definitions of \endPlancktable and \endPlancktablewide for A&A style notes 
% set to one-column  or full-page width, respectively
\def\endPlancktable{\tablewidth=\columnwidth 
    $$\hss\copy\tablebox\hss$$
    \vskip-\lastskip\vskip -2pt}
\def\endPlancktablewide{\tablewidth=\textwidth 
    $$\hss\copy\tablebox\hss$$
    \vskip-\lastskip\vskip -2pt}
\def\tablenote#1 #2\par{\begingroup \parindent=0.8em
    \abovedisplayshortskip=0pt\belowdisplayshortskip=0pt
    \noindent
    $$\hss\vbox{\hsize\tablewidth \hangindent=\parindent \hangafter=1 \noindent
    \hbox to \parindent{$^#1$\hss}\strut#2\strut\par}\hss$$
    \endgroup}
\def\doubleline{\vskip 3pt\hrule \vskip 1.5pt \hrule \vskip 5pt}

%-----------------------------------------------------------------------
% useful macros
%-----------------------------------------------------------------------
%
\def\nside{\ifmmode N_{\mathrm{side}}\else $N_{\mathrm{side}}$\fi}
%
\def\L2{\ifmmode L_2\else $L_2$\fi}
%
\def\dtt{\Delta T/T}
\def\DeltaT{\ifmmode \Delta T\else $\Delta T$\fi}
\def\deltat{\ifmmode \Delta t\else $\Delta t$\fi}
\def\fknee{\ifmmode f_{\rm knee}\else $f_{\rm knee}$\fi}
\def\Fmax{\ifmmode F_{\rm max}\else $F_{\rm max}$\fi}
%
\def\solar{\ifmmode{\rm M}_{\mathord\odot}\else${\rm M}_{\mathord\odot}$\fi}
\def\Msolar{\ifmmode{\rm M}_{\mathord\odot}\else${\rm M}_{\mathord\odot}$\fi}
\def\Lsolar{\ifmmode{\rm L}_{\mathord\odot}\else${\rm L}_{\mathord\odot}$\fi}
%
\def\inv{\ifmmode^{-1}\else$^{-1}$\fi}
\def\mo{\ifmmode^{-1}\else$^{-1}$\fi}
\def\sup#1{\ifmmode ^{\rm #1}\else $^{\rm #1}$\fi}
\def\expo#1{\ifmmode \times 10^{#1}\else $\times 10^{#1}$\fi}
%
\def\,{\thinspace}
\def\lsim{\mathrel{\raise .4ex\hbox{\rlap{$<$}\lower 1.2ex\hbox{$\sim$}}}}
\def\gsim{\mathrel{\raise .4ex\hbox{\rlap{$>$}\lower 1.2ex\hbox{$\sim$}}}}
\let\lea=\lsim
\let\gea=\gsim
\def\simprop{\mathrel{\raise .4ex\hbox{\rlap{$\propto$}\lower 1.2ex\hbox{$\sim$}}}}
%
\def\deg{\ifmmode^\circ\else$^\circ$\fi}
\def\pdeg{\ifmmode $\setbox0=\hbox{$^{\circ}$}\rlap{\hskip.11\wd0 .}$^{\circ}
          \else \setbox0=\hbox{$^{\circ}$}\rlap{\hskip.11\wd0 .}$^{\circ}$\fi}
\def\arcs{\ifmmode {^{\scriptstyle\prime\prime}}
          \else $^{\scriptstyle\prime\prime}$\fi}
\def\arcm{\ifmmode {^{\scriptstyle\prime}}
          \else $^{\scriptstyle\prime}$\fi}
\newdimen\sa  \newdimen\sb
\def\parcs{\sa=.07em \sb=.03em
     \ifmmode \hbox{\rlap{.}}^{\scriptstyle\prime\kern -\sb\prime}\hbox{\kern -\sa}
     \else \rlap{.}$^{\scriptstyle\prime\kern -\sb\prime}$\kern -\sa\fi}
\def\parcm{\sa=.08em \sb=.03em
     \ifmmode \hbox{\rlap{.}\kern\sa}^{\scriptstyle\prime}\hbox{\kern-\sb}
     \else \rlap{.}\kern\sa$^{\scriptstyle\prime}$\kern-\sb\fi}
%
\def\ra[#1 #2 #3.#4]{#1\sup{h}#2\sup{m}#3\sup{s}\llap.#4}
\def\dec[#1 #2 #3.#4]{#1\deg#2\arcm#3\arcs\llap.#4}
\def\deco[#1 #2 #3]{#1\deg#2\arcm#3\arcs}
\def\rra[#1 #2]{#1\sup{h}#2\sup{m}}
%
\def\page{\vfill\eject}
\def\dots{\relax\ifmmode \ldots\else $\ldots$\fi}
%
%-----------------------------------------------------------------------
% units
%-----------------------------------------------------------------------
%
\def\WHzsr{\ifmmode \,\mathrm{W\,Hz\mo\,sr\mo}\else \,W\,Hz\mo\,sr\mo\fi}
\def\MHz{\ifmmode \,\mathrm{MHz}\else \,MHz\fi}
\def\GHz{\ifmmode \,\mathrm{GHz}\else \,GHz\fi}
\def\mKs{\ifmmode \,\mathrm{mK\,s}^{1/2}\else \,mK\,s$^{1/2}$\fi}
\def\muKs{\ifmmode \,\mu\mathrm{K\,s}^{1/2}\else \,$\mu$K\,s$^{1/2}$\fi}
\def\muKRJs{\ifmmode \,\mu\mathrm{K_{RJ}}\,\mathrm{s}^{1/2}\else \,$\mu$K$_{\mathrm{RJ}}$\,s$^{1/2}$\fi}
\def\muKRJ{\ifmmode \,\mu\mathrm{K_{RJ}}\else \,$\mu$K$_{\mathrm{RJ}}$\fi}
\def\muKCMB{\ifmmode \,\mu\mathrm{K_{CMB}}\else \,$\mu$K$_{\mathrm{CMB}}$\fi}
\def\KRJ{\ifmmode \,\mathrm{K_{RJ}}\else \,K$_{\mathrm{RJ}}$\fi}
\def\KCMB{\ifmmode \,\mathrm{K_{CMB}}\else \,K$_{\mathrm{CMB}}$\fi}
\def\muKHz{\ifmmode \,\mu\mathrm{K\,Hz}^{-1/2}\else \,$\mu$K\,Hz$^{-1/2}$\fi}
\def\MJysr{\ifmmode \,\mathrm{MJy\,sr\mo}\else \,MJy\,sr\mo\fi}
\def\MJysrmK{\ifmmode \,\mathrm{MJy\,sr\mo}\,\mathrm{mK}_{\mathrm{ CMB}}\mo\else \,MJy\,sr\mo\,mK$_{\mathrm{CMB}}\mo$\fi}
\def\microns{\ifmmode \,\mu\mathrm{m}\else \,$\mu$m\fi}
\def\micron{\microns}
\def\muK{\ifmmode \,\mu\mathrm{K}\else \,$\mu$\hbox{K}\fi}
\def\microK{\ifmmode \,\mu\mathrm{K}\else \,$\mu$\hbox{K}\fi}
\def\muW{\ifmmode \,\mu\mathrm{W}\else \,$\mu$\hbox{W}\fi}
\def\kms{\ifmmode \,\mathrm{km\,s}^{-1}\else \,km\,s$^{-1}$\fi}
\def\kmsMpc{\ifmmode \,\kms\,\mathrm{Mpc\mo}\else \,\kms\,Mpc\mo\fi}
\def\cmisq{\ifmmode \,\mathrm{cm}^{-2}\else $\,\mathrm{cm}^{-2}$\fi}
%

%-----------------------------------------------------------------------
% Parameters
%-----------------------------------------------------------------------
\def\Acmb{\ifmmode a_\mathrm{CMB}\else $a_{\mathrm{CMB}}$\fi}
\def\Aquad{\ifmmode a_\mathrm{quad}\else $a_{\mathrm{quad}}$\fi}
\def\Asynch{\ifmmode a_\mathrm{s}\else $a_{\mathrm{s}}$\fi}
\def\Asrc{\ifmmode a_\mathrm{src}\else $a_{\mathrm{src}}$\fi}
\def\Adust{\ifmmode a_\mathrm{d}\else $a_{\mathrm{d}}$\fi}
\def\Asdust{\ifmmode a_\mathrm{sd}\else $a_{\mathrm{sd}}$\fi}
\def\Aame{\ifmmode a_\mathrm{ame}\else $a_{\mathrm{ame}}$\fi}
\def\Aco{\ifmmode a_\mathrm{CO}\else $a_{\mathrm{CO}}$\fi}
\def\AcoOne{\ifmmode a_\mathrm{CO10}\else $a_{\mathrm{CO10}}$\fi}
\def\AcoTwo{\ifmmode a_\mathrm{CO21}\else $a_{\mathrm{CO21}}$\fi}
\def\AcoThree{\ifmmode a_\mathrm{CO32}\else $a_{\mathrm{CO32}}$\fi}
\def\Aff{\ifmmode a_\mathrm{ff}\else $a_{\mathrm{ff}}$\fi}
\def\Tcmb{\ifmmode T_\mathrm{CMB}\else $T_{\mathrm{CMB}}$\fi}
\def\Tdust{\ifmmode T_\mathrm{d}\else $T_{\mathrm{d}}$\fi}
\def\scmb{\ifmmode s_\mathrm{CMB}\else $s_{\mathrm{CMB}}$\fi}
\def\squad{\ifmmode s_\mathrm{quad}\else $s_{\mathrm{quad}}$\fi}
\def\ssynch{\ifmmode s_\mathrm{s}\else $s_\mathrm{s}$\fi}
\def\sdust{\ifmmode s_\mathrm{d}\else $s_{\mathrm{d}}$\fi}
\def\ssdust{\ifmmode s_\mathrm{sd}\else $s_{\mathrm{sd}}$\fi}
\def\same{\ifmmode s_\mathrm{ame}\else $s_{\mathrm{ame}}$\fi}
\def\ssrc{\ifmmode s_\mathrm{src}\else $s_{\mathrm{src}}$\fi}
\def\sco{\ifmmode s_\mathrm{CO}\else $s_{\mathrm{CO}}$\fi}
\def\sff{\ifmmode s_\mathrm{ff}\else $s_{\mathrm{ff}}$\fi}
\def\gff{\ifmmode g_\mathrm{ff}\else $g_{\mathrm{ff}}$\fi}
\def\fsynch{\ifmmode f_\mathrm{s}\else $f_{\mathrm{s}}$\fi}
\def\fsd{\ifmmode f_\mathrm{sd}\else $f_{\mathrm{sd}}$\fi}
\def\fame{\ifmmode f_\mathrm{ame}\else $f_{\mathrm{ame}}$\fi}
\def\alphasrc{\ifmmode \alpha_\mathrm{src}\else $\alpha_{\mathrm{src}}$\fi}
\def\bdust{\ifmmode \beta_\mathrm{d}\else $\beta_{\mathrm{d}}$\fi}
\def\bsynch{\ifmmode \beta_\mathrm{s}\else $\beta_{\mathrm{s}}$\fi}
\def\bsun{\ifmmode \beta_\mathrm{sun}\else $\beta_{\mathrm{sun}}$\fi}
\def\nuzeros{\ifmmode \nu_{0,\mathrm{s}}\else $\nu_{0,\mathrm{s}}$\fi}
\def\nuzeroff{\ifmmode \nu_{0,\mathrm{ff}}\else $\nu_{0,\mathrm{ff}}$\fi}
\def\nuzerod{\ifmmode \nu_{0,\mathrm{d}}\else $\nu_{0,\mathrm{d}}$\fi}
\def\nuzeroame{\ifmmode \nu_{0,\mathrm{ame}}\else $\nu_{0,\mathrm{ame}}$\fi}
\def\nuzerosd{\ifmmode \nu_{0,\mathrm{}}\else $\nu_{0,\mathrm{sd}}$\fi}
\def\nuzerosrc{\ifmmode \nu_{0,\mathrm{src}}\else $\nu_{0,\mathrm{src}}$\fi}
\def\nup{\ifmmode \nu_{\mathrm{p}}\else $\nu_{\mathrm{p}}$\fi}
\def\alphasd{\ifmmode \alpha_{\mathrm{sd}}\else $\alpha_{\mathrm{sd}}$\fi}
\def\Te{\ifmmode T_{\mathrm{e}}\else $T_{\mathrm{e}}$\fi}
\def\lmax{\ifmmode \ell_{\mathrm{max}}\else $\ell_{\mathrm{max}}$\fi}
\def\NHI{\ifmmode N_{\mathrm{H\,\textsc i}}\else $N_{\mathrm{H\,\textsc i}}$\fi}
\def\chisq{\ifmmode \chi^2\else $\chi^2$\fi}


%---------------------------------------------------------------
% constants
%---------------------------------------------------------------
\def\kB{\ifmmode k_\mathrm{B}\else $k_{\mathrm{B}}$\fi}
%
%
%----------------------------------------------------------------------
% set up machinery to list Planck papers in roman numeral order.
%----------------------------------------------------------------------

\providecommand{\sorthelp}[1]{}



\def\WMAP{\textit{WMAP}}
\def\COBE{\textit{COBE}}
\def\LiteBIRD{\textit{LiteBIRD}}
\def\LCDM{$\Lambda$CDM}
\def\ffp{FFP6}
\def\unionmask{U73}
\def\nside{N_{\mathrm{side}}}

\def\healpix{\texttt{HEALPix}}
\def\commander{\texttt{Commander}}
\def\commanderone{\texttt{Commander1}}
\def\commandertwo{\texttt{Commander2}}
\def\commanderthree{\texttt{Commander3}}
\def\ruler{\texttt{Ruler}}
\def\comrul{\texttt{Commander-Ruler}}
\def\CR{\texttt{C-R}}
\def\nilc{\texttt{NILC}}
\def\gnilc{\texttt{GNILC}}
\def\sevem{\texttt{SEVEM}}
\def\smica{\texttt{SMICA}}
\def\CamSpec{\texttt{CamSpec}}
\def\Plik{\texttt{Plik}}
\def\XFaster{\texttt{XFaster}}

\renewcommand{\d}[0]{\vec{d}}
\renewcommand{\t}[0]{\vec{t}}
\newcommand{\A}[0]{\tens{A}}
%\newcommand{\Y}[0]{\tens{Y}}
\newcommand{\Y}[0]{\tens{Y}}
\newcommand{\n}[0]{\vec{n}}
\newcommand{\red}[0]{\color{red}}
\newcommand{\green}[0]{\color{green}}
\newcommand{\s}[0]{\vec{s}}
\renewcommand{\a}[0]{\vec{a}}
\newcommand{\m}[0]{\vec{m}}
\newcommand{\f}[0]{\vec{f}}
\newcommand{\F}[0]{\tens{F}}
\newcommand{\B}[0]{\tens{B}}
\newcommand{\T}[0]{\tens{T}}
\newcommand{\Cp}[0]{\tens{C}}
\renewcommand{\L}[0]{\tens{L}}
\newcommand{\g}[0]{\vec{g}}
\newcommand{\N}[0]{\tens{N}}
\newcommand{\M}[0]{\tens{M}}
\newcommand{\iN}[0]{\tens{N}^{-1}}
\newcommand{\iM}[0]{\tens{M}^{-1}}
\newcommand{\w}[0]{\vec{w}}
\renewcommand{\S}[0]{\tens{S}}
\renewcommand{\r}[0]{\vec{r}}
\renewcommand{\u}[0]{\vec{u}}
\newcommand{\q}[0]{\vec{q}}
\renewcommand{\v}[0]{\vec{v}}
\renewcommand{\P}[0]{\tens{P}}
\newcommand{\dt}[0]{d_t}
\newcommand{\di}[0]{d_i}
\newcommand{\nt}[0]{n_t}
\newcommand{\st}[0]{s_t}
\newcommand{\mt}[0]{m_t}
\newcommand{\ft}[0]{f_t}
\newcommand{\Te}[0]{T_{\rm e}}
\newcommand{\EM}[0]{\rm EM}
\newcommand{\mathsc}[1]{{\normalfont\textsc{#1}}}
\newcommand{\hi}{\ensuremath{\mathsc {Hi}}}

\newcommand{\BP}{\textsc{BeyondPlanck}}
\newcommand{\cosmoglobe}{\textsc{Cosmoglobe}}
\newcommand{\npipe}[0]{\texttt{NPIPE}}
\newcommand{\sroll}[0]{\texttt{SROLL}}
\newcommand{\srollTwo}[0]{\texttt{SROLL2}}
\newcommand{\HEALPix}[0]{\texttt{HEALPix}}

\def\bC{\tens{C}}
\def\ba{\vec{a}}
\def\ncha{N_\mathrm{cha}}
\def\nfg{N_\mathrm{fg}}

\newcommand{\mbeam}{\ensuremath{m_b}}
\newcommand{\mbmax}{\ensuremath{m_{b\text{,max}}}}
\newcommand{\cvar}{\ensuremath{c(\vartheta, \varphi, \psi)}}
\newcommand{\atom}{\ensuremath{\text{\textbf{A2M}}}}

\newcommand{\includegraphicsdpi}[3]{
    \pdfimageresolution=#1  % Change the dpi of images
    \includegraphics[#2]{#3}
    \pdfimageresolution=72  % Change it back to the default
}

\renewcommand{\topfraction}{1.0}	% max fraction of floats at top
    \renewcommand{\bottomfraction}{1.0}	% max fraction of floats at bottom
    %   Parameters for TEXT pages (not float pages):
    \setcounter{topnumber}{2}
    \setcounter{bottomnumber}{2}
    \setcounter{totalnumber}{4}     % 2 may work better
    \setcounter{dbltopnumber}{2}    % for 2-column pages
    \renewcommand{\dbltopfraction}{0.9}	% fit big float above 2-col. text
    \renewcommand{\textfraction}{0.04}	% allow minimal text w. figs
    %   Parameters for FLOAT pages (not text pages):
    \renewcommand{\floatpagefraction}{0.9}	% require fuller float pages
	% N.B.: floatpagefraction MUST be less than topfraction !!
    \renewcommand{\dblfloatpagefraction}{0.9}	% require fuller float pages

\def\adj{^{\dagger}}
\def\tp{^{\rm T}}
\def\inv{^{-1}}

\begin{document}

\title{\bfseries{\scshape{BeyondPlanck}} VIII. Efficient Sidelobe Convolution and Corrections through Spin Harmonics}
\input{BP10_authors.tex}
\authorrunning{Galloway et. al.}
\titlerunning{Sidelobe Corrections}

\abstract{We introduce a new formulation of the \texttt{Conviqt} convolution algorithm in terms of spin harmonics, and apply this to the problem of sidelobe correction for \BP, the first end-to-end Bayesian Gibbs sampling framework for CMB analysis. We compare our implementation to the previous \Planck\ LevelS implementation, and find good agreement between the two codes in terms of accuracy, but with a speed-up reaching a factor of 3--10, depending on the frequency bandlimits, $l_{\textrm{max}}$ and $m_{\textrm{max}}$. The new algorithm is significantly simpler to implement and maintain, since all low-level calculations are handled through an external spherical harmonic transform library. We find that our mean sidelobe estimates for \Planck\ LFI agree well with previous efforts. Additionally, we present novel sidelobe rms maps that quantify the uncertainty in the sidelobe corrections due to variations in the sky model.}

\keywords{ISM: general -- Cosmology: observations, polarization,
    cosmic microwave background, instrument characterization -- Galaxy:
    general}

\maketitle


\section{Introduction}
\label{sec:introduction}

One of the important systematic effects that must be accounted for in CMB instruments is telescope stray light and sidelobes \citep[e.g.,][]{barnes2003,planck2013-p02a}. This is the non-zero response of the detector to areas of the sky outside the main beam, however that is defined. Because microwave telescopes typically work near their diffraction limit, some level of sidelobes is inevitable. Furthermore, their structure can be complicated by many different physical effects, such as spurious optical reflections or manufacturing irregularities in the detectors or optical elements. These signal contributions can have far-reaching consequences on the observed signal, particularly at large angular scales, as they do not behave in the same sky-stationary manner as the main beam signal.

Sidelobe signals can produce many types of errors in CMB analysis pipelines, and they represent a potent source of systematic contamination \citep[e.g.,][]{planck2014-a04,bp17}. In particular, as the sidelobe response functions often are broadly distributed, this contamination can confuse important signals like the CMB solar and orbital dipoles that are used for calibration. Sidelobes uncertainties couple directly with foreground emission from diffuse galactic components, producing an important source of contamination. In some experiments a further contaminating signal can originate from a source not on the sky, such as ground pickup or radio-frequency (RF) noise. In all cases, sidelobe signal is detrimental to the quality of the final sky maps and parameter estimates, and requires a dedicated effort to remove it. Characterizing and correcting these spurious signals is therefore an important part of optimal CMB mapmaking, and requires optimized algorithms to characterize them efficiently. One of the most important of these is to convolve a beam or sidelobe response function with a sky map or model to generate a re-observed map. 

Full sky convolution on the sphere is a problem that has been important in the CMB field since the earliest satellite measurements. Earlier experiments like \COBE\ \citep{cobe_sl} and \WMAP\ \citep{barnes2003} either did not model sidelobes or used simple pixel-based convolution approaches which even for their low resolution required radially symmetric beam approximations \citep{radialapprox} or limited the applications to large scales \citep{burigana2001}.

\citet{Wandelt:2001} presented the first harmonic space convolution algorithm, often referred to as ``total convolution'', which achieved a large performance gain ($O\left(\sqrt{N_\mathrm{pix}}\right)$) over the pixel-based methods ($O\left(N_\mathrm{pix}\right)$). This initial breakthrough allowed the calculation of these convolutions easily enough that they could be applied to each simulation, instead of requiring a dedicated study necessitating months of runtime. 


Next, \citet{conviqt} developed the \texttt{Conviqt} approach, which was used in several official \Planck\ analysis pipelines (\citealt{planck2014-a04}, \citealt{planck2014-a10}, \citealt{npipe}) and to generate the \Planck\ Full Focal Plane (FFP) simulations \citep{planck2014-a14}. This approach was an improvement over the state of the art, speeding up the computation of the Wigner recursion relationships used in their harmonic space algorithm, as well as providing a standardized, user friendly library, \texttt{libconviqt}, that was incorporated into numerous pipelines. 

In this paper, we introduce a new formulation of the \texttt{Conviqt} algorithm that is based on Spherical Harmonic Transforms (SHTs), rather than directly computing the Wigner matrix elements. We are thus able to leverage the highly optimized \texttt{libsharp} SHT library to perform the bulk of the calculations \citep{libsharp}. Although this new approach was not developed specifically for \BP, this paper is the first to explicitly derive, discuss, and benchmark the method.

\section{Sidelobes, \texttt{libconviqt} and \texttt{libsharp}}

\subsection{Total Convolution through Spin Harmonics}

\label{sec:conviqt}

Given a sky map, $s(\hat{n})$, and beam, $b(\hat{n})$, our task is to
compute a quantity $\cvar \in \mathbb{R}$ that represents the
convolution of these two fields, with the beam oriented in polar
coordinates $(\vartheta, \varphi)$\footnote{In this paper, $\vartheta$ and $\varphi$ are the co-latitude and longitude of a location on the sphere, i.e., they have the same meaning as in the \healpix\ context \citep{gorski2005}.}, and rotated around its own central
axis by $\psi$,
\begin{equation}
\begin{aligned}
  \cvar &:= s_{lm_s} * b_{lm_b}\\ 
   &\equiv \int_{4\pi} s(\hat{n})
  b\big(\hat{n}'(\vartheta,\varphi)-\hat{n},\psi\big)\, d\Omega_{\hat{n}}.
  \label{eq:convolution}
\end{aligned}
\end{equation}
Here, $s_{lm_s}$ denotes the spherical harmonic coefficients of the sky signal, and $b_{lm_b}$ is the beam in the same representation. Care has been taken to distinguish between $m_s$ ($m_\text{sky}$) and $m_b$ ($m_\text{beam}$), as the two indices will be treated separately in the following derivation.

A computationally efficient solution for this problem was derived by
\citet{conviqt}, who exploited fast recurrence relations for Wigner
$d$ matrix elements to evaluate Eq.~\eqref{eq:convolution} in harmonic
space. In the following, we will show that this equation
can alternatively be expressed in terms of spin-harmonics. The
resulting algebra is in principle identical to the recursion relations
used by \citet{conviqt}, but the implementation is simply repackaged
in a format that is significantly easier to implement in practical
computer code, since it may use existing and highly optimized
spherical harmonic libraries, such as \texttt{libsharp} \citep{libsharp}, to perform
the computationally expensive parts.

As shown by \citet{Wandelt:2001}, Eq.~(\ref{eq:convolution}) can be
evaluated efficiently in harmonic space as
\begin{equation}
\cvar = \sum_{l,m_s,m_b} s_{lm_s} b^\ast_{lm_b}
[D^{l}_{m_sm_b}(\varphi,\vartheta,\psi)]^\ast,
\label{eq:totalconvolver}
\end{equation}
where $s_{lm_s}$ and $b_{lm_b}$ are the spherical harmonic
coefficients of the signal and beam, respectively, and
$D^{l}_{m_sm_b}$ is the Wigner $D$-matrix.

This may be expressed as \citep{goldberg}
\begin{equation}
D^l_{-ms}(\varphi,\vartheta,-\psi) = (-1)^m\sqrt{\frac{4\pi}{2l+1}}
{}_sY_{lm}(\vartheta,\varphi) e^{is\psi},
\end{equation}
where ${}_sY_{lm}(\vartheta,\varphi)$ is the spin-weighted
spherical harmonic and the placement of the negative signs are an arbitrary historical convention. Inserting this expression into
Eq.~(\ref{eq:totalconvolver}) yields
\begin{equation}
\cvar = \sum_{l,m_s,m_b}\sqrt{\frac{4\pi}{2l+1}} s_{lm_s} b_{l-m_b}\, \cdot {}_{-m_b}Y_{lm_s}(\vartheta,\varphi) e^{im_b\psi},
\end{equation}
where we have assumed that the beam is real-valued in position space,
and we have used the symmetry relations,
\begin{align}
  D^l_{-m_s,-m_b}(\vartheta,\varphi)&=(-1)^{m_s+m_b}[D^{l}_{m_s,m_b}(\vartheta,\varphi)]^\ast,\\
  b^\ast_{l,m_b} (-1)^{m_b}&=b_{l,-m_b}.\label{eq:breal}
\end{align}
Pulling the summation over $m_b$ in front of the other sums yields
\begin{equation}
\cvar = \sum_{m_b} e^{im_b\psi} \sum_{l,m_s} \sqrt{\frac{4\pi}{2l+1}}
s_{lm_s} b_{l-m_b}\, \cdot {}_{-m_b}Y_{lm_s}(\vartheta,\varphi).
\label{eq:c_m}
\end{equation}
The terms in this outer sum can be arranged in the form $m_b=0, \pm 1, \pm 2, \dots$.
The contribution from \mbox{$m_b=0$} can be interpreted as a spin-0
spherical harmonic transform of the quantity $\sqrt{4\pi/(2l+1)}
s_{lm_s} b_{l0}$, which can be easily computed by a library like
\texttt{libsharp} \citep{libsharp}.

Since $\cvar \in \mathbb{R}$, we know that the contributions from the pairs
$m_b=\pm 1, \pm 2, \dots$ must be complex conjugate with respect to each other, and their combined
contribution is therefore 
\begin{align}
e^{im_b\psi}{}_{m_b}&S(\vartheta,\varphi) + e^{-im_b\psi}{}_{m_b}S^\ast(\vartheta,\varphi) =
 \\\nonumber
 &2\left[\cos(m_b\psi)\text{Re}({}_{m_b}S(\vartheta,\varphi)) -
 \sin(m_b\psi)\text{Im}({}_{m_b}S(\vartheta,\varphi))\right],
\end{align}
where we have defined
\begin{equation}
{}_{m_b}S(\vartheta,\varphi) \equiv \sum_{l,m_s} \sqrt{\frac{4\pi}{2l+1}} s_{lm_s}
b_{l-m_b}\, \cdot {}_{-m_b}Y_{lm_s}(\vartheta,\varphi).
\label{eq:S}
\end{equation}
This is a spherical harmonic transform of a quantity with spin $m_b$,
which can also be computed efficiently by \texttt{libsharp}.

In practice, the transforms in Eq.~(\ref{eq:S}) are implemented by
separating $S$ into its gradient and curl (or $E$ and $B$)
coefficients, $a_{lm}$ \citep{lewis_2005}, 
\begin{equation}
{}_{m_b}S_{lm_s} = -\left({}_{m_b}E_{lm_s} + i\,{}_{m_b}B_{lm_s}\right),
\end{equation}
using the symmetry relations
${}_{m_b}E_{l-m_s}=(-1)^{m_s}{}_{m_b}E_{lm_s}^*$ and
${}_{m_b}B_{l-m_s}=(-1)^{m_s}{}_{m_b}B_{lm_s}^*$, and the overall minus sign is a convention. Again making use of the symmetry relation
in Eq.~\eqref{eq:breal}, this results in
\begin{align}
  {}_{m_b}E_{l,m_s} &= -s_{lm_s} \text{Re}(b_{l,\mbeam}) \label{spin1}\\
  {}_{m_b}B_{l,m_s} &= -s_{lm_s} \text{Im}(b_{l,\mbeam}) \label{spin2}.
\end{align}

To summarize, efficient evaluation of the convolution integral in
Eq.~(\ref{eq:convolution}) may be done through the following steps:
\begin{enumerate}
  \item For each $m = {0,\ldots, m_{b}}$, pre-compute the spin spherical
    harmonic coefficients in Eqs.~\eqref{spin1}--\eqref{spin2}, and
    compute the corresponding spin-$m_{b}$ SHT
    with an external library such as \texttt{libsharp}; this results
    in a three-dimensional data cube of the form
    $c(\vartheta,\varphi,m_{b})$.
  \item For each position on the sky, $(\vartheta,\varphi)$, perform a
    Fourier transform to convert these coefficients to
    $c(\vartheta,\varphi,\psi)$, as given by Eq.~\eqref{eq:c_m}.
\end{enumerate}
In practice, the resulting $c(\vartheta,\varphi,\psi)$ data object is
evaluated at a finite pixel resolution typically set to match the beam
bandlimit. To obtain smooth estimates within this data object, a wide
range of interpolation schemes may be employed, trading off
computational efficiency against accuracy. This issue is
identical to previous approaches \citep{Wandelt:2001,conviqt},
and we refer the interested reader to those papers for further details.

\subsection{Comparison with \texttt{libconviqt}}

To compare the results of this new total convolution approach with the older \texttt{libconviqt} approach of \citet{conviqt}, we evaluate the convolution between the beam for one of the LFI 30 GHz receivers (28M) and a \commander\ 30 GHz sky model \citep{bp13,bp14} using both methods. The resulting convolution cubes are then observed using LFI's scanning strategy for the first year of the \Planck\ flight. The resulting map differences are shown in Fig.~\ref{fig:differences}. The convolution cubes are also directly compared for accuracy, and found to agree with an integrated difference at the $10^{-8}$ level, indicative of differences at the level of numerical precision.

\begin{figure}[t]
  \center
  \includegraphics[width=\linewidth]{scripts/conviqt_diff_map.pdf}
  \caption{Map level difference in temperature of the new SHT convolution algorithm compared to the old \texttt{Conviqt} approach, observed using the identical pointing of the first year of the 30 GHz \Planck\ detectors. The differences are at the level of machine precision, indicating full agreement between the two algorithms. 
  }\label{fig:differences}
\end{figure}

Fig.~\ref{fig:speed} compares the runtime between the two approaches for a test configuration with an elliptical beam and a fixed sky model, and with $m_\mathrm{max}=0$ (a radially symmetric beam) and $m_\mathrm{max}=10$, respectively. In both cases, the new implementation outperforms the old approach at all but the lowest $l_\mathrm{max}$, where the data read time dominates. Additionally, for compatibility with the old \texttt{libconviqt} approach, this test was performed with an older version of \texttt{libsharp}, so we expect that the new algorithm scales even more favourably than this with the latest implementation. We note that this is a significant real-life advantage of the new approach: any improvement in SHT libraries, which typically are subject to intensive algorithm development and code maintenance, translates directly into a computational improvement for the convolution algorithm. 


\begin{figure}[t]
  \center
  \includegraphics[width=\linewidth]{scripts/runtime.pdf}
	\caption{Runtime comparison between the \texttt{libconviqt} approach and the new spin-SHT approach for the convolution of an elliptical Gaussian with a set of random sky $a_{l,m}$s. This work ties or outperforms the previous approach for all values of $l_\mathrm{max}$ from 256 to 8192 for both $m_\mathrm{max}$ values shown. Note the log scale on the y-axis.
  }\label{fig:speed}
\end{figure}

\section{Sidelobe Models}

Fig.~\ref{fig:slresponse} shows characteristic sidelobe response functions evaluated at a fixed frequency on the sky for a detector in each Planck LFI band. The sidelobe response for each detector within a single \Planck\ band look visually quite similar, so only these representative ones are shown here. Each is stored on disk as a set of $a_{lm}$'s with $l_\mathrm{max} = 512$ and $m_\mathrm{max} = 100$. 

\begin{figure*}[t]
  \center
  \includegraphics[width=0.33\linewidth]{scripts/sl_27_M.pdf}
  \includegraphics[width=0.33\linewidth]{scripts/sl_24_M.pdf}
  \includegraphics[width=0.33\linewidth]{scripts/sl_18_M.pdf}\\
  \caption{Maps of the sidelobe response on the sky from a representative detector at (left to right) 30\,GHz, 44\,GHz and 70\,GHz. The beam orientation is such that the main beam is pointed directly at the north pole in these maps. The intensities are normalized such that the main beams have unit power at $l=0$.
  }\label{fig:slresponse}
\end{figure*}

\subsection{Main Beam Treatment}

In the \BP\ analysis, the sidelobe and main beam components of the sky response are separated, and the sidelobes are treated as a nuisance signal similar to the orbital dipole correction term, as can be seen in the \BP\ global parametric model of the data:
\begin{equation}
\begin{split}
d_{j,t} = g_{j,t}&\left[\P_{tp,j}\B_{pp',j}\sum_{c}
\M_{cj}(\beta_{p'}, \Delta bp^{j})a^c_{p'}  + s^{\mathrm{orb}}_{j,t}  
+ s^{\mathrm{fsl}}_{j,t}\right] + \\
+ &n^{\mathrm{corr}}_{j,t} + n^{\mathrm{w}}_{j,t}.
\end{split}
\label{eq:datamodel}
\end{equation}
The other terms in this equation are discussed in detail in \cite{BP01}, but here the main beam signal is denoted as $B_{pp',j}$ and the sidelobe signal is extracted from the signal contribution and expressed as $s^{\mathrm{fsl}}_{j,t}$. This distinction allows the sidelobes to be treated separately from the main beam in all respects. Treating the main beam using the \texttt{Conviqt} formalism of this paper would be possible, but the additional precision needed to model it accurately would require much higher $l_\mathrm{max}$, and therefore greatly increased computational time and memory requirements. 

In the \BP\ analysis, the main beam is used (in conjunction with the sidelobes) to compute the full 4$\pi$ dipole response, as detailed in Sect.~\ref{sec:dipole}. Additionally, a Gaussian main beam approximation is used during component separation to smooth the sky model to the appropriate beam resolution for each channel. During mapmaking, beam effects are ignored and the beam is assumed to be pointed at the center of each pixel. 

\subsection{Sidelobe Normalization}
\label{sec:normalization}

The normalization of the sidelobes differs slightly from the normalization used within the \Planck\ LFI collaboration. The \Planck\ 2018 LFI beam products leave a small portion (around 1\%) of known missing power within the system unassigned due to uncertainties about to which component it should be assigned \citep{planck2014-a05}. In the current analysis, we rather adopt the same approximation as for \Planck\ DR4 \citep{npipe}, and renormalize the beam transfer function such that this power is distributed proportionally at each $l$; that is, we rescale the beam transfer function $B_l$ such that its full sky integral is $B_0 = 1$. This re-scaling is equivalent to assigning the unknown beam power uniformly over the entire beam. We note that this normalization is in either case always done before any higher-level analysis for both \Planck\ 2018 and DR4; the only difference is whether the renormalization must be performed by external users through deconvolution of a non-unity normalized main beam transfer function or not. 

\subsection{Orbital Dipole and Quadrupole Sidelobe Response}
\label{sec:dipole}

\begin{figure*}[t]
  \center
  \includegraphics[width=0.33\linewidth]{scripts/030_I_sl_mean.pdf}
  \includegraphics[width=0.33\linewidth]{scripts/030_Q_sl_mean.pdf}
  \includegraphics[width=0.33\linewidth]{scripts/030_U_sl_mean.pdf}\\
    \includegraphics[width=0.33\linewidth]{scripts/044_I_sl_mean.pdf}
  \includegraphics[width=0.33\linewidth]{scripts/044_Q_sl_mean.pdf}
  \includegraphics[width=0.33\linewidth]{scripts/044_U_sl_mean.pdf}\\
    \includegraphics[width=0.33\linewidth]{scripts/070_I_sl_mean.pdf}
  \includegraphics[width=0.33\linewidth]{scripts/070_Q_sl_mean.pdf}
  \includegraphics[width=0.33\linewidth]{scripts/070_U_sl_mean.pdf}\\
  \caption{Maps of the sidelobes convolved with the sky at each of the three LFI frequencies. From top to bottom: 30\,GHz, 44\,GHz and 70\,GHz. The left column is the unpolarized sky signal, the central column is the Q polarization and the right column is U. Note the difference in the colour scales required to see the same level of detail in all three channels. 
  }\label{fig:slmean}
\end{figure*}

The treatment of the sidelobes is also important while generating orbital dipole and quadrupole estimates. Because \Planck\ is calibrated primarily from the dipole measurements \citep{planck2016-l01,npipe,BP07}, the sidelobe's contribution to the dipole can directly result in an absolute calibration error if not handled appropriately. While the CMB Solar dipole can easily be handled using the \texttt{Conviqt} approach described in Sec.~\ref{sec:conviqt}, the orbital dipole is not sky-stationary and thus must be handled separately. 

\BP\ generates orbital dipole and quadrupole estimates directly from the \Planck\ pointing information, using the satellite velocity data which has been stored at low resolution (one measurement per pointing period). With this information, it is possible to estimate the orbital dipole and quadrupole amplitude for each timestep, allowing the time-domain removal of the signal before it contaminates the final products with non-sky-stationary signal artifacts. Additionally, once this signal has been isolated from the raw data, it can be used as an aid in the calibration routines because of its highly predictable structure.


\BP\ uses the same technique as \Planck\ DR4 \citep[see appendix C]{npipe} to generate the orbital dipole and quadrupole estimate. That is, we express the signal $\tilde{D}$ seen by a detector observing a fixed direction $\hat{\boldsymbol n}_0$ as the convolution of the dipole and quadrupole signal on the sky, $D(\hat{\boldsymbol n})$, with the full $4\pi$ beam response, $B(\hat{\boldsymbol n}, \hat{\boldsymbol n}_0)$,
\begin{equation}
\tilde{D}(\hat{\boldsymbol n}_0) = \int d\Omega\, B(\hat{\boldsymbol n}, \hat{\boldsymbol n}_0) D(\hat{\boldsymbol n}).
\end{equation}
Here, it is useful to break the dipole signal up into three orthogonal components in the standard Cartesian coordinates $(x,y,z)$, and we adopt the convention that the main beam points toward the north pole in our coordinate system.

The orbital CMB dipole and quadrupole can be expressed as a Doppler shift in each direction \citep{Notari:2015},
\begin{equation}
D(\hat{\boldsymbol n}) = T_0\left[ \beta \cdot \hat{\boldsymbol n}(1 + q\boldsymbol  \beta \cdot \hat{\boldsymbol n}) \right],
\label{eq:dipole_simple}
\end{equation}
where $\beta$ is the satellite velocity divided by the speed of light $\beta = \frac{\mathrm{v}}{c}$, $T_0$ is the CMB temperature and $q$ is quadrupole factor dependent on the frequency $\nu$, defined by 
\begin{equation}
q = \frac{a}{2} \frac{e^a + 1}{e^a -1}, \ \ \  \textrm{where} \ \ \ a = \frac{h\nu}{k_B T_0}.
\end{equation}
Inserting these expressions into Eq.~(\ref{eq:dipole_simple}), one obtains
\begin{equation}
\begin{split}
\tilde{D} = T_0 \int d\Omega_{\hat{\boldsymbol n}} B(\hat{\boldsymbol n}, \hat{\boldsymbol n}_0) \left[ x \ \beta_x + y\ \beta_y +  z\ \beta_z +\right. \\
q\left(x^2\ \beta^2_x + y^2\ \beta^2_y + z^2\ \beta^2_z + \right. \\
\left. \left. 2xy\ \beta_x\beta_y + 2xz\ \beta_x\beta_z + 2yz\ \beta_y\beta_z\right) \right],\label{eq:dipole}
\end{split}
\end{equation}
where $\hat{\boldsymbol n} = (x,y,z)$ is a unit direction vector that is also the integration variable, $\hat{\boldsymbol n}_0$ is the fixed direction of the satellite pointing for this timestep. Noting that the geometric factors in this expression may be precomputed as
\begin{equation}
S_{x} = \int x\, B(\hat{\boldsymbol n}, \hat{\boldsymbol n}_0)\, d\Omega_{\hat{\boldsymbol n}},
\end{equation}
we see that Eq.~\eqref{eq:dipole} may be written in the following form,
\begin{align}
\tilde{D} &= T_0 \left[ S_x \ \beta_x + S_y\ \beta_y +  S_z\ \beta_z +\right. \nonumber\\
  &\quad\quad\,\,\,\, q\left(S_{xx}\ \beta^2_x + S_{yy}\ \beta^2_y + S_{zz}\ \beta^2_z + \right. \nonumber\\
  &\quad\quad\quad\,\,\left.\left.2S_{xy}\ \beta_x\beta_y + 2S_{xz}\ \beta_x\beta_z + 2S_{yz}\ \beta_y\beta_z\right) \right]\label{eq:dipole_fast}.
\end{align}
To compute $\tilde{D}$ for an arbitrary beam orientation, one simply needs to rotate the satellite pointing and velocity vectors into the coordinate system used to define $S$, and then one can evaluate Eq.~\eqref{eq:dipole_fast} very quickly. 

\begin{figure*}[t]
  \center
  \includegraphics[width=0.33\linewidth]{scripts/030_I_sl_rms.pdf}
  \includegraphics[width=0.33\linewidth]{scripts/030_Q_sl_rms.pdf}
  \includegraphics[width=0.33\linewidth]{scripts/030_U_sl_rms.pdf}\\
    \includegraphics[width=0.33\linewidth]{scripts/044_I_sl_rms.pdf}
  \includegraphics[width=0.33\linewidth]{scripts/044_Q_sl_rms.pdf}
  \includegraphics[width=0.33\linewidth]{scripts/044_U_sl_rms.pdf}\\
    \includegraphics[width=0.33\linewidth]{scripts/070_I_sl_rms.pdf}
  \includegraphics[width=0.33\linewidth]{scripts/070_Q_sl_rms.pdf}
  \includegraphics[width=0.33\linewidth]{scripts/070_U_sl_rms.pdf}\\
  \caption{Sidelobe rms maps at each of the three LFI frequencies. From top to bottom: 30\,GHz, 44\,GHz and 70\,GHz. The left column is the unpolarized sky signal, the central column is the Q polarization and the right column is U. Again note the different colour scales.
  }\label{fig:slrms}
\end{figure*}

\BP\ further accelerates this operation by computing this rotation for only one point in twenty (chosen so as to still fully sample the dipole), and using a spline to interpolate between them. This saves the cost of calculating a new rotation matrix at each step, and instead relies on the smoothness of the signal to ensure continuity. The algorithm treats the final few points of each pointing period that do not divide evenly into the subsampling factor separately. This allows the use of regular bin widths, which greatly speeds up the splining routines, while the final few points are calculated using the slower rotation matrix technique. 

\begin{figure}[t]
  \center
  \includegraphics[width=0.95\linewidth]{figs/030_diff.pdf}\\
  \includegraphics[width=0.95\linewidth]{figs/044_diff.pdf}\\
  \includegraphics[width=0.95\linewidth]{figs/070_diff.pdf}
  \caption{Frequency map difference plots smoothed to one degree at (top to bottom) 30, 44 and 70\,GHz, comparing two pipeline executions with the same seed, one of which has no sidelobe correction.
  }\label{fig:freqdiff}
\end{figure}

\begin{figure}[t]
  \center
  \includegraphics[width=0.95\linewidth]{figs/cmb_diff.pdf}\\
  \includegraphics[width=0.95\linewidth]{figs/synch_diff.pdf}\\
  \includegraphics[width=0.95\linewidth]{figs/freefree_diff.pdf}\\
  \includegraphics[width=0.95\linewidth]{figs/ame_diff.pdf}
  \caption{Component map difference plots for (top to bottom) CMB, synchrotron, AME and freefree emission, comparing two pipeline executions with the same seed, one of which has no sidelobe correction.
  }\label{fig:compdiff}
\end{figure}

\section{Sidelobe Estimates}
\subsection{Posterior mean corrections}

Fig.~\ref{fig:slmean} shows the mean sidelobe signal estimates at each of the three LFI frequencies for the entire mission, co-added across each frequency and projected into sky coordinates, identically to the way the true sky signal is treated. Each map is averaged over 90 Gibbs samples produced in the main \BP\ analysis \citep{BP01}, after discarding burn-in and thinning the remaining chain by a factor of 10.

We note that these maps follow the traditional \Planck\ LFI method of sidelobe correction by producing these signals in the time domain during TOD processing. These templates are therefore exactly correct for the maps produced by these pipeline runs, but will not match precisely with analyses that use different data cuts, flagging or channel selection.

These results look visually similar to the corresponding \Planck\ DPC results presented in Fig.~7 of \citet{planck2014-a04}. The main difference is that the current results also include the CMB dipole, whereas the LFI 2015 DPC analysis showed the sidelobe pickup of dipole-subtracted maps. We see that the sidelobe signal is strongest at 30\,GHz, and that the dominant features in the co-added sky maps consist of a series of rings created by the interplay between the sidelobe pickup and bright Galactic plane features. 

Fig. \ref{fig:slmean} also clearly indicates that the overall level of sidelobe pickup at 44 GHz is significantly lower than for the 30 and 70 GHz channels. This is due to the particular location in the focal plane of two of the three 44 GHz feedhorns, which results in a significant under-illumination of both the primary and secondary reflectors of the Planck telescope for those two horns (see Figure 4 of \cite{sandri2010}).

\subsection{Error Propagation}

In addition to the posterior mean sidelobe maps, the \BP\ pipeline outputs also provide an estimate of the sidelobe stability and statistical variation. Fig.~\ref{fig:slrms} shows the rms maps generated from the same sample of sidelobe signal estimates as was used in Fig.~\ref{fig:slmean}. Clear evidence of the scanning pattern can be seen, which is expected. The sharp vertical lines visible in polarization (clearest in 30 GHz Q and U at the top, and 44 GHz $U$ at the top and bottom) have been previously examined by the \Planck\ team, and are caused by a chance alignment between the non-dense \Planck\ scanning strategy and the shape of the \healpix\ pixels. For an example of this effect, see Fig.~15 of \citet{planck2013-p03c}.

These posterior rms maps cannot be considered true sidelobe error estimates, however, as they do not account for uncertainties in the sidelobe response itself. Rather, they only show the change in the estimated sidelobe signal due to sky model variations from component separation. Full sidelobe error propagation would require sampling over the physical parameters that determine the detectors' sidelobe response on the sky. While sampling the full set of optical model parameters is likely to be infeasible due to excessive computational time, identifying a minimal parameter set that may account for the main potential variations in the sidelobe response functions, and precomputing response functions over a grid of such parameters, would result in physically motivated uncertainties for the sidelobe models. This approach will be developed for future applications such as the \LiteBIRD\ mission \citep{LiteBIRD}.

\section{Impact on Frequency and Component Maps}

To assess the importance of sidelobe corrections on frequency and component maps, we perform two runs of the  \commander\ code, starting from the same input data as the main pipeline run, and with identical random seeds. As a comparison, we remove the far sidelobe correction from one of these secondary pipeline executions, and we difference the results between these two pipelines. 

Fig.~\ref{fig:freqdiff} shows the differences in the frequency maps between the two cases in temperature, where the effects are the most obvious. The only large-scale features that can clearly be seen are the small dipole differences (most clearly visible at 70\,GHz). These are directly caused by the dipolar component seen in Fig.~\ref{fig:slmean}, as this contribution to the total sky signal that was in the sidelobe term is now unaccounted for. In previous analyses, these dipole contributions were handled through specific modeling of exactly these effects, but this test makes it explicitly clear that correct dipole measurements require accurate knowledge of the sidelobe pickup.

Second, we see two more features in the difference maps that are more localized. The first of these are the ring structures that match the actual sidelobe map structures quite closely. These are of course the same rings from Fig.~\ref{fig:slmean}, which are not corrected for in the second pipeline run (where the sidelobe corrections are omitted). Additionally, there are some uniform residuals that are visible in the Galactic plane regions of the difference maps. These are caused by calibration mismatch between the detectors at a single frequency. As each of the detectors now sees a slightly different dipole signal on the sky, depending on its specific sidelobe response, their calibrations do not agree with one another, which causes signal residuals which are most visible in the plane where the signal amplitude is highest.

Fig.~\ref{fig:compdiff} shows the differences in component maps from this same comparison, again in temperature. The CMB as well as the three low-frequency foreground components are estimated using the standard \commanderthree\ technique described in \cite{bp13}. The AME component sees similar issues to the ones seen by the frequency maps above. The dipole is slightly incorrect, there are sidelobe-esque stripes and the Galactic plane shows a strong residual, all of which are effects that have been seen directly in the frequency maps. This dipole difference seen here is precisely the one that contributes to the difference in calibration between the two different pipeline executions. 

The other three components (synchrotron, CMB and free-free) show less structural difference when compared. They have absorbed some of the sidelobe-like ring structures, but the primary difference can be seen most clearly in the Galactic plane. Here, we notice a large residual caused by the inaccurate model of the Galactic emission being altered slightly by the gain and calibration differences between the two runs. As the Galactic emission is significantly brighter than the rest of the sky, small changes in calibration produce large errors like the ones seen here.

\begin{figure}[t]
  \center
  \includegraphics[width=0.95\linewidth]{figs/ncorr_diff.pdf}\\
  \caption{Difference in correlated noise, projected into the map domain at 30 GHz comparing two pipeline executions with the same seed, one of which has no sidelobe correction.
  }\label{fig:ncorr}
\end{figure}

Finally, in Fig.~\ref{fig:ncorr}, we see the correlated noise map difference at 30\,GHz between the two runs. Here, we see that some of the missing sidelobe signal has been accommodated by the correlated noise component. These structures mirror the strongest sidelobe-like signals in the 30\,GHz difference map in the top panel of Fig.~\ref{fig:freqdiff}, showing that we can absorb some of these sidelobe differences into the noise. This is in fact helpful, as it allows for some leeway in the final sidelobe model, as small uncertainties and artifacts that are inconsistent between different frequency channels will be mostly absorbed into the correlated noise component, rather than in the sky maps. The differences that we see in the maps, however, indicate that this process is not perfect, as some of the spurious signal still makes it to the final maps without a perfect sidelobe model. For a real-world example of these issues, we refer the interested reader to the on-going \BP\ re-analysis of the \WMAP\ data, for which far sidelobe contamination appears to be a dominant problem \citep{bp17}.

The residual errors seen in Figs.~\ref{fig:freqdiff} and \ref{fig:compdiff} are also present in the \BP\ analysis, albeit at much lower levels. We know that our knowledge and modeling of the sidelobes are imperfect, as they are based on limited measurements of the physical LFI sidelobes, and some of the power is unaccounted for. Future applications of the pipeline that aim for a robust $r\le 0.01$ measurement will be required to marginalize over the sidelobe uncertainties in some manner, either by directly Gibbs sampling a subset of the instrument parameters or by parameterizing and fitting sidelobe error estimates. We do not believe that the sidelobe contribution causes significant errors in the LFI sample sets produced by \BP, as it is unlikely to be more than a 10\,\% error on the sidelobe estimates of Fig. \ref{fig:slmean}. At 30 GHz, this corresponds to at most a 0.05\,\% error in our temperature maps and a 1\,\% error in polarization. We do expect however, that as instrumental sensitivities improve, especially in polarization, this sidelobe term will need to be modeled very accurately, and the corresponding uncertainties must be propagated properly, for instance using methods similar to those presented in this paper.

\section{Summary and Conclusions}
\label{sec:conclusions}

This paper presents a formulation of the \texttt{Conviqt} algorithm in terms of spin-weighted spherical harmonics. This algorithm is already implemented in the latest versions of the \texttt{libconviqt} library, and it has now also been re-implemented directly into \commander, where it is used for sidelobe corrections for the \BP\ analysis framework. Based on the Monte Carlo samples produced in that analysis, we have presented novel posterior mean and standard deviation maps for each of the three \Planck\ LFI frequency bands.

The full-sky sidelobe treatment techniques presented here are easily generalizable to other experiments, and can be tuned to match the required spatial characteristics of other instruments simply by adjusting the spherical harmonic bandpass parameters, $l_\mathrm{max}$ and $m_\mathrm{max}$, of the sidelobe description. The only requirement for using the code with a new instrument is a \healpix-compatible description of the sidelobe response function per detector. The more accurate this characterization of the instrument is, the better the sidelobe estimate will approximate the true sidelobe contamination in the timestream.

We note that the approach presented here is less useful for ground or balloon based experiments where the dominant sidelobe pickup contains radiation from an environmental source. This pickup is not sky-synchronous, and thus cannot be modeled purely as a beam-sky convolution, but must include additional contributions from, for example, telescope baffles, ground pickup or clouds. For these types of experiments, other techniques such as aggressive baffling are likely better suited.

We also stress that the current implementation only supports sidelobe error propagation for sky model uncertainties, not uncertainties in the actual sidelobe response function itself, and these are very likely to dominate the total sidelobe error budget. Future work should therefore aim to introduce parametric models for the sidelobe response itself, and sample (or, at least, marginalize) over the corresponding free parameters, as these are typically one of the most important unknowns for many experiments.

Finally, we note that future CMB experiments such as \LiteBIRD, which are targeting low $B$-mode limits, may need to consider more complex ways of handling sidelobes and beams. The ultimate solution in this respect is $4\pi$ beam convolution for every single timestep, which could be achieved using a similar framework to the approach discussed here. This would remove the sidelobes as a nuisance signal from the data model of Eq.~(\ref{eq:datamodel}) and instead incorporate them directly into the beam term, $\B_{pp',j}$. This approach should be feasible for a relatively low-resolution experiment such as \LiteBIRD, and will be investigated going forward.

\begin{acknowledgements}
  We thank Prof.\ Pedro Ferreira for useful suggestions, comments and
  discussions, and Ms.\ Diana Mjaschkova-Pascual for administrative
  support. We also thank the entire \Planck\ and \WMAP\ science teams
  for invaluable support and discussions, and for their dedicated
  efforts through several decades without which this work would not be
  possible. The current work has received funding from the European
  Union’s Horizon 2020 research and innovation programme under grant
  agreement numbers 776282 (COMPET-4; \BP), 772253 (ERC;
  \textsc{bits2cosmology}), and 819478 (ERC; \textsc{Cosmoglobe}). In
  addition, the collaboration acknowledges support from ESA; ASI, CNR,
  and INAF (Italy); NASA and DoE (USA); Tekes, AoF, and CSC (Finland);
  RCN (Norway); ERC and PRACE (EU).
\end{acknowledgements}


\bibliographystyle{aa}

\bibliography{../common/Planck_bib,../common/BP_bibliography,sources}

\end{document}
